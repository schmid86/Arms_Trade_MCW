\documentclass[12pt, letterpaper]{article}

 \usepackage{helvet}
 \usepackage[english]{babel}
 \usepackage{float}
 \usepackage{url}
 \usepackage[labelfont=bf]{caption}
 \usepackage{longtable, array, ragged2e}
	\newcolumntype{L}[1]{>{\RaggedRight\hspace{0pt}}p{#1}}
 \usepackage{amssymb,amsmath,amsthm}
 \usepackage{color}
 \usepackage{setspace}
 \usepackage{graphicx}
 \usepackage{natbib}
	\setcitestyle{notesep={: }}
 \usepackage{fancyvrb}
 \usepackage{enumitem}
 \usepackage{changepage}
\usepackage{setspace}
\doublespacing
 \usepackage{rotating}
 \DeclareMathOperator{\aVar}{aVar}
 \renewcommand{\theequation}{\thesection.\arabic{equation}}
 \numberwithin{equation}{section}
 \renewcommand{\bibsection}{}

\usepackage{fancyhdr}
\pagestyle{fancy}
\fancyhf{}
\fancyhead[R]{\thepage}
%\fancyhead[L]{Network Interdependencies and the Evolution of Arms Trade 1953-2013}
\let\oldthebibliography=\thebibliography
\let\endoldthebibliography=\endthebibliography
\renewenvironment{thebibliography}[1]{%
   \begin{oldthebibliography}{#1}%
     \setlength{\itemsep}{-.0ex}%
}%
{%
   \end{oldthebibliography}%
}

\begin{document}


%\renewcommand{\BBN}{:\;} % muss nach \begin{document} stehen

\title{Network Interdependencies and the Evolution of the International Arms Trade
%\footnote{G\"oran Kauermann and Paul W. Thurner gratefully acknowledge the support of the German Research Foundation (KA 1188 10-1, TH 697/9-1). Additionally, SJC gratefully acknowledges the support of the National Science Foundation (SES-1357622, SES-1461493, SES-1514750) and the Alexander von Humboldt Foundation. }
}
\author{Paul W. Thurner\footnote{Institute of Political Science, Ludwig Maximilians University Munich (LMU)}\and Christian S. Schmid\footnote{Department of Statistics, The Pennsylvania State University} \and Skyler J. Cranmer\footnote{Department of Political Science, The Ohio State University}  \and G\"oran Kauermann\footnote{Department of Statistics, Ludwig Maximilians University Munich}}
%\date{Word count: 11,433}



\maketitle

\spacing{1}

    
% ABSTRACT
%--------------------------------------------------------

\begin{abstract}
\doublespacing
\noindent Since few states are able to produce all of their own military hardware, a majority of countries’ military systems rely on weapon imports. The structure of the international defense technology exchange network remains an important puzzle to understand, along with the factors that drive its evolution. Drawing on a political economy model of arms supply, we propose a new network-oriented explanation for the world-wide transactions of major conventional weapons in the period after World War II. Using temporal exponential random graph models, our dynamic approach illustrates how network dependencies and the relative weighting of economic versus security considerations vary over time. One of our major results is to demonstrate how security considerations started regaining importance over economic ones after 2001. Additionally, our model exhibits strong out-of-sample predictive performance, with network dependencies contributing to model improvement especially after the Cold War.
\end{abstract}
    
\begin{center}
Word count Abstract: 140 \\[0.1cm]
Word count total: 10,883 (incl. footnotes, figures, tables and references)
\end{center}    
% INTRODUCTION
%--------------------------------------------------------
\newpage

 \section{Introduction}
\doublespacing
%%%% Intro Paragraph
% 2-3 sentences introducing problem
The international trade of arms is an essential precondition for import-reliant countries to bolster their military might. Exporting countries benefit twofold from these transactions since the orders stimulate their domestic military industries and the produced goods support foreign allied countries and governments. It remains a puzzle how this global defense technology exchange network is structured, and which causal factors drive its evolution. These are important questions, particularly because it is often argued that the patterns of international arms transfers have changed rapidly in recent years, with Western countries generally reducing their military budgets in the aftermath of the 2007/2008 global financial crises, and Asian and African countries increasing their expenditures and imports dramatically \citep[][175-182]{SIPRI:14}. The end of the Cold War is considered to have put commercial aspects in the foreground of suppliers' considerations, and thereby to have increased the complexity of arms trade relations as compared to previous structures clearly defined along security-based alignments. 
The question arises, therefore, of whether is it for reasons of initiating defense alliances or strengthening existing ones that superpowers provide weapons, as in the case of massive transfers from the USSR to India and from the US to Saudi Arabia?
Or are business considerations paramount - as could be concluded from cases where supply countries sell at the same time to allies and to foes of allies - more recently illustrated by Russia's military sales to Armenia and Azerbaijan?
In order to test these interesting conjectures on the changing structures and dynamics of the international arms trade empirically, we propose a network theory of arms transfers and offer appropriate new statistical tools for its examination. These contributions are of enormous practical and theoretical value because arms transactions constitute an essential part of the ''structure of international security'' \citep{BuzaWaev:03} potentially influencing local and regional power shifts.       
 
% 3-6 sentences on what we do; emphasis on novelty and on puzzle stated by the political economy approach
We conceive of international arms transfers as comprising a dynamic global network that is influenced by both economic forces and geopolitical power dynamics. The potential trade-off between these two mechanisms over time is unknown. Moreover, we posit that the structural embeddedness of these transfers is more relevant than previously supposed. More precisely, we expect once-established trade relations to exhibit very strong path dependencies. Market concentration among a handful of supplier countries leads to a hierarchical dependence structure. Furthermore, we expect dyadic arms trades to be situated in subsystemic, hyperdyadic configurations, i.e. in so-called nested hierarchical structures involving more than just the dyads under consideration. We show that developed supplier countries frequently 'share' import countries. 
  
%  bow to existing lit
While there is an extensive body of literature on the international arms trade (see for overviews \cite{Krau:92}, \cite{GarcLevi:07}), there are few quantitative studies providing generalizable insights and even fewer treatments of the system that recognize its full relational complexity \citep{AkerSeim:14, Kinne:2016}. To fill this gap we propose a new quantitative network approach for the implementation of an established political economy model of arms trade.
% summary of what we do
Our aim is to describe, explain, and even predict the structure and dynamics of the international arms trade network from after World War II to present day (1950 - 2013). For this purpose, we utilize temporal exponential random graph models, which are the new standard for statistical analysis of dynamic networks. \citep{HaFuXi:10, Rtergm, LeiCraDes:17}.

Our empirical analysis reveals that genuine network processes -- particularly path-dependencies, importer and exporter effects, hierarchical three-country configurations, but also economic and security-related factors play a decisive, but time-varying role in the formation and evolution of the arms trade network. 
 
Overall, our model predicts the arms trade network quite well, with an out-of-sample predictive accuracy consistently over 90\% by the ROC criteria and between 40\% to 60\% by the precision-recall criteria. We observe a sharp decline in prediction performance after the end of the Cold War. The Post-Cold War  period is obviously characterized by a rewiring of the network -  the impact of often discussed network dependencies is especially clear here.    
Most importantly, we find that geopolitical alliances become progressively less important as determinants of arms trade activity after the Cold War but regain influence after 2001. This study provides the first quantitative corroboration of previous conjectures that a new international regime of security cooperation has emerged in recent years. 

    
    
% LIT REVIEW
%--------------------------------------------------------

\section{International Arms Trade: The State of Knowledge} 


% Theoretical innovations
Much of the literature on arms trade is based on historical analyses \citep{Hark:75, Krau:92}, case studies  \citep{Kolo1987, Gill1992}, and descriptive analyses of arms exportation to developing states \citep{Leisetal:70,  BrzoOhls:87}. 
The field achieved important progress in theory building when \cite{Levietal:94} introduced an economic model of buying countries, selling countries, and companies within the selling states. This supply and demand model conceives of arms trade as an imperfect market where only a few countries are able to develop advanced military technologies. The oligopolistic structure in the Post-WWII period has been evident for the informed observer. However, its dynamic emergence and persistence continue to be a puzzle. The structure was created and perpetuated by strong governmental intervention and regulation, research and development subsidies, support via domestic procurement, and export promotions. Thus, market entry based on autochthonously developed products proved to be extremely difficult for other countries. From the perspective of the importing countries, the need for interoperable components, long-term maintenance contracts, and customization of the military personnel led to a strong adherence to one or few supply country/countries. This market is peculiar as transactions are not always compensated by regular payment. Contracts can also specify price reductions, loans, offsets, military aid, or even barter deals.  

In order to capture these specificities, the model proposed by \citet{Levietal:94} includes two arguments in the objective function of governments: supply countries simultaneously maximize economic returns and minimize future negative security externalities from weapon sales. The importing countries maximize security by building up weapon stock piles while minimizing economic strains. Currently, it is completely unclear how this presumed optimization process is carried out in reality. Especially with regard to the hypothesized economics versus security trade-off for supplier countries, it would be essential to quantify this relationship. However, there are no empirical results on this aspect of arms trade. 

Overall, the quantitative literature on arms trade is quite small and has tended to focus either on importing \citep{Pear:89, SmitTasi:05} or exporting \citep{Blan:05, PerkNeum:10} countries. Few studies have attempted to integrate the two perspectives in an encompassing network study design. 
As of this writing, exactly two studies have considered the network structure of international arms trade at the global level. 
\cite{AkerSeim:14} were the first to propose a network perspective for visualizing the transfers of major conventional weapons, and to provide a descriptive analysis for the period 1950-2007. 
Their estimation approach covers the period 1962-2000 and is based on a gravity model including GDP per capita, geographic distance, and regime similarity.
The authors use a pooled linear probability model with country-fixed effects to explain the binary occurrence of trade relations between two countries. A main result is that the 'implied relative likelihood of trade' was conditional on regime dissimilarity during the Cold War, but not after the dissolution of the USSR. Note that this study assumes dyadic independence, i.e. all dyads are independent, including those involving the same country. 
A recent article \citep{Kinne:2016} comes closest to treating arms trade as a network and relaxing the dyadic independence assumption by accounting for dependence structures beyond dyads. The author runs a stochastic actor-oriented model (SIENA) \citep{art:Snijders2002} for the recent period 1995-2010. He uses a two-stage estimation model in which the formation of weapons cooperation agreements is captured in the first stage. The second stage explains volumes of arms transfer activities (being transformed into rank orders) of a given country $i$, where activities are defined as the sum of exports and imports of a specific country under consideration. 
Explanatory covariates of this equation are limited to attributes of the respective country.  
A major finding of this article is that weapons cooperation agreements and weapons trade are strongly interrelated. 

Despite contributing valuable theoretical and empirical insights into international arms trade, the existing literature has yet to treat it as a dynamic network. Importantly, we lack a proper network theory of arms trade. We also need empirical evidence on genuine network effects, on changes of effects over time in a true dynamic modeling approach, and on the predictive power of the implemented models. This is what we intend to provide in this article using a temporal exponential random graph model approach.


 
%%%%%%%%%%%%%%%%%%%%%%%%%%%%%%%%%%%%%%%%%%%%%%%%%%
%%%%%%%%%%%%%%%%%%%%%%%%%%%%%%%%%%%%%%%%%%%%%%%%%%
%%%%%%%%%%%%%%%%%%%%%%%%%%%%%%%%%%%%%%%%%%%%%%%%%%
%%%%%%%%%%%%%%%%%%%%%%%%%%%%%%%%%%%%%%%%%%%%%%%%%%
\section{\large Trade of Major Conventional Arms: Descriptive Explorations from a Network Perspective}
 \subsection{SIPRI Data}
 
% What the data are and what they cover
The most comprehensive database of arms transfers available is that compiled by the Stockholm International Peace Research Institute (SIPRI). 
We use SIPRI data on the exchange of major conventional weapons (MCW) between 182 countries for the period 1950-2013. 
We excluded a small number of microstates and non-sovereign territories and organizations, see the Appendix for details. MCW refer to aircraft, air defense systems, anti-submarine systems, armored vehicles, artillery, missiles, satellites, ships, etc. \citep{SIPRI:14}. 
All voluntary sales, gifts, and loans of MCW between all countries, including manufacturing licenses, are also included. 

To measure the volume of international arm transfers, SIPRI has developed a unique monitoring system. 
Transfer data are given as \textit{trend indicator values} (TIV), a measure based on production costs, not representing sales prices (SIPRI 2014: 271-272).
This measure has the considerable advantage of being consistent over time, thus making it possible to compare arms flows across time and space. 


To analyze the network of international arms transfers over the period of observation, we construct a time-series of annually measured arms trade networks. We follow \citet{AkerSeim:14} in coding our dependent variable numerically as one if there is a transfer from country A to country B, zero otherwise. In accordance with SIPRI's coding this means that if the value of deliveries is more than 500 000 USD, then, a value of 1 is assigned. 
In the Appendix we provide additional material showing that this thresholding does not bias the content of our results. 



 \subsection{Descriptive Results}
% \begin{figure}[t]
%\begin{center}
%\includegraphics[scale=.46]{numberactors_0}
%\caption{The number of actors included in the arms trade networks (top) and the density of the networks (bottom) for the time period 1950-2013}
%\label{numberactors}
%\end{center}
%\end{figure}
Visualizing the binarized arms transfer networks, we see a bipolar core-periphery structure until the 1990s, with the US and the USSR constituting the centers of supply (see Appendix 3.1). Thus, the often proclaimed East-West polarization and the segmentation into two components is apparent in the arms trade dimension in the early 1950s, but then disappears gradually. The successive participation of more and more countries and the overall increase of transfer activities leads to such a complexity that fine-grained structures and processes are no longer detectable from pure visual inspection. 

%Therefore, it is unclear whether trade relations are still aligned or whether political block structures directed by the two super powers softened.\\
%\indent We provide first a short descriptive analysis of the underlying network structures. 
%The majorities of countries that are actively involved in the arms trade network are only receiving weapons and are not selling weapons to other countries. This leads to star-like patterns with peripheral satellite countries. For the sake of clarity, the countries that did not trade weapons in all in these years were excluded from the plots. %Overall about 95\% of the edges (trades) are one-sided with little changes over the years. Thus, intra-sectoral trade in the form of reciprocal edges in arm trading is rare.
%The top plot in Figure \ref{numberactors} shows the number of actors (states) in the network for each year. Due to decolonization, there is a conspicuous and constant growth of actors from the 1960s until 1980. Another jump results from the break-up of the USSR in 1990/1991. In the bottom plot we visualize the time series of the network's density.

\begin{figure}
\begin{center}
\includegraphics[scale=.35]{aggregated_volumes}
\caption{Aggregated Trade Volumes in billion SIPRI Trend Indicator Values 1950-2013}
\label{tiv_agg}
\end{center}
\end{figure}

%The density of a network is defined as the number of actual ties in a network divided by the number of possible ties. Here we have to be careful when interpreting the trend because the number of actors changes over time. We observe a peak in 1982, followed by a decrease until the nineties. %Note also, that there is a first peak in the end of 1950s. 
%As the number of countries did hardly change in the final 15-20 years of the series, it can be concluded that the density of the networks rose steadily since the begin of the new millennium, with the exception of 2013. Trade frequency is on the rise in the last years.





\begin{figure}
\begin{center}
\includegraphics[scale=.37]{reciprocity_0}
\caption{Proportion of one-sided ties (red) to mutual ties (blue) over the time period 1950-2013 }
\label{reciprocity}
\includegraphics[scale=.45]{degree_distribution}
\caption{Proportions of indegrees and outdegrees for the time period 1950-2013. For each degree value, 90\% of the occurring values is situated between the whiskers  }
\label{deg_dist}
\end{center}
\end{figure} 

In Figure \ref{tiv_agg} we show the total transfer value for each year. Its decline since the early 1980s (1980-1984) proved to be transient and turned into a renewed increase with the beginning of the new millennium (i.e. a ca. 30 percent rise through 2013). \citet[241]{SIPRI:14} indicates 14 percent growth rates for the period 2009-2013 as compared to the previous five year period, with the US, Russia, Germany, China, and France as the top 5 exporters (see \citet[258]{SIPRI:14}). The appearance of China in the group of the top 5 exporters starting in 2012 foreshadows its emergence as a major military power. 
The hub-periphery structure becomes also manifest in Figure \ref{reciprocity}. Here, we see that around 95\% of all ties are one-directional. Thus, intra-sectoral trade in the form of reciprocated edges in the arms trade network is quite rare. 
To give an example which can be directly compared to the analysis of the commercial trade network by \cite{WaAhAr:13}: in 2008 our network contains 22 mutual, 315 asymmetric, and 29075 null relations given 172 nodes as compared to 8169 mutual dyads, 2896 asymmetric dyads and about 5000 null relations in the commercial trade network. This illustrates the much higher extent of the presence of the reciprocity mechanism in general commercial trade in a globalized world as compared to arms trade.

Finally, Figure \ref{deg_dist} presents the distribution of indegrees (number of inwards directed links, i.e. import relations) and outdegrees (number of outwards directed links, i.e. export relations) from 1950 to 2013 as percentages.
Plotting the distribution on a percentage scale allows a comparison of the distributions for yearly networks with a different number of actors. 90\% of the relevant degree values are situated within the black bars. Almost 80\% of all actors in the network have an outdegree of zero, i.e. the majority of countries actually do not export military hardware in a country-year. Performing a test of whether outdegrees follows a power-law distribution (see \cite{Clauset:2009}) and thus indicating market concentration, we see that, with the exception of 1959 and 1962, in all years the market activity in terms of trade frequency follows a power law. The most active traders in 2008 are the US, with 51 relations, followed by Germany and France, with 30 relations each. 
This pronounced market concentration is not met in the commercial trade market (see \cite{WaAhAr:13}).    

On the other hand, more than 30 percent of the countries have an indegree of $0$. This means that even after excluding microstates that exhibit no MCW at all, a very large number of the remaining countries do not actively trade weapons every single year. Note too that a high proportion of countries only purchase weapons from one supplier. These importers usually do not sell weapons themselves and are dependent on a single supplier. Countries with more than eight degrees (suppliers) in 2008 include, e.g.: Pakistan (12), USA (11), Chile, Poland, Indonesia, Brazil (each 9). 


While it is common in network analysis to hypothesize dependency structures that go beyond dyads, e.g. the closures of three nodes (triads), we know of no political science contribution where the actual country compositions of typical triadic configurations in the data have been identified. In the following we provide real world examples of such closures by carrying out a count of all triads occurring in our network\footnote{Note that in the case of directed networks sixteen triads are possible \citep{DavisLeinhardt:72}. For an overview see chapter 14 in \cite{Wasserman:94}.}. 

Table \ref{triad2} presents one of the most frequent patterns of triadic closures summed over the whole period of observation. Over time we see a steady increase of these triadic closures. This corroborates our theoretical expectation of transitivity processes indicating the ''friend-of-a-friend-is-my-friend'' relationship. Additionally, the realized compositions point to an interesting pattern of two countries backing a third, just as captured by the configuration of edge-wise shared partners \citep{HunHan:06, SnijPattRobHand:06}.  Considering Table \ref{triad2} the US and the UK frequently take the position of A or B, and they often share an identical partner C (e.g. Japan, Thailand, Saudi Arabia, Turkey, etc.), but also engage in mutual exchange. The second most frequent partner in the B position of the US (if the US is in the A position) is France. Our selection of the most frequent yearly triadic configurations illustrates important network processes of closing triads and sharing or ''sandwiching'' partners. They allow for the creation of trust beyond dyads, but also for the joint control of third parties.

 % \begin{table}[p]

  %\scriptsize

%\center{\normalsize Most Frequent Triadic Configurations and Edgewise Shared Partners}
%\caption{ $\underset{ A \rightarrow B  \leftarrow C ; A \rightarrow C}{\protect\includegraphics[height=1.4cm]{030t.png}}$}
%\label{triad1}
%\centering
%\begin{tabular}{rlllr}


%  \hline
% & A & B & C & Freq \\ 
%  \hline
%1 & United States & Pakistan & Sweden &  32 \\ 
%  2 & France & Pakistan & China &  29 \\ 
%  3 & Germany & Pakistan & China &  25 \\ 
%  4 &  United States & Turkey & Germany &  24 \\ 
%  5 &  United States & Denmark & Sweden &  23 \\ 
%  6 & Germany &Egypt & China &  22 \\ 
%  7 &  United States & Greece & Germany &  21 \\ 
%  7 & Russia & Iran & China &  21 \\ 
%  9 &  United States & Spain & Germany &  20 \\ 
%  9 &  Germany & Turkey & Italy &  20 \\ 
%  11 & United States & Brazil & Sweden &  19 \\ 
%  11 &  United States & Greece & Netherlands &  19 \\ 
%  11 &  France & Spain & Italy &  19 \\ 
%  11 & France & India & Italy &  19 \\ 
%  15 & United States & Turkey & Netherlands &  18 \\ 
%  15 & France & India & United Kingdom &  18 \\ 
%  15 &  France & Japan & Italy &  18 \\ 
%  18 & United Kingdom & Japan & Italy &  17 \\ 
%  18 & United States & Spain & Italy &  17 \\ 
%  18 & United States & Denmark & Germany &  17 \\ 
%   \hline
%\end{tabular}
%\caption*{$A \rightarrow B  \leftarrow C ; A \rightarrow C$}
%\end{table}

\begin{table}[t]
 \scriptsize
\caption{ $\underset{A \rightarrow B  \leftarrow C ; A \leftrightarrow C}{\protect\includegraphics[height= 1.4cm]{120u.png}}$}
\label{triad2}
\centering
\begin{tabular}{rlllr}
  \hline
 & A & B & C & Freq \\ 
  \hline
1 & United Kingdom & Japan & United States &  35 \\ 
  1 & United Kingdom & Saudi Arabia & United States &  35 \\ 
  3 & United Kingdom & Thailand & United States &  32 \\ 
  4 & United Kingdom & Turkey & United States &  29 \\ 
  4 & United Kingdom & India & United States &  29 \\ 
  6 & United Kingdom & Brazil & United States &  28 \\ 
  7 & United Kingdom & Denmark & United States &  27 \\ 
  8 & United Kingdom & Chile & United States &  26 \\ 
  8 & United States & Turkey & Italy &  26 \\ 
  10 & United States & Pakistan & France &  24 \\ 
  10 & United Kingdom & United Arab Emirates & United States &  24 \\ 
  10 & United States & Brazil & Canada &  24 \\ 
  10 & United States & Belgium & France &  24 \\ 
  14 & United States & Netherlands & France &  23 \\ 
  14 & United States & Turkey & Germany &  23 \\ 
  14 & United States & Greece & France &  23 \\
  17 & United States & India & France &  22 \\ 
  17 & United States & Turkey & France &  22 \\ 
  17 & United States & Greece & Germany &  22 \\ 
  17 & United States & Brazil & Italy &  22 \\ 
   \hline
\end{tabular}

\end{table}
In sum, already our exploratory descriptive analysis indicates strong network dependency structures - i.e. a network topology with strong outdegree centralization, a pronounced non-reciprocity, and the frequent presence of triadic transitive closures where two export countries share third import countries. These potential network dependencies call for a statistical modeling approach which goes beyond dyadic independence, i.e. the assumption that the occurrence of relationship between two countries A and B is independent from the occurrence of any other dyadic relationship C and D, or even of A and D. This assumption is obviously unrealistic.  

\section{A Network Theory of Arms Trade }\label{hypo}

We propose to consider the international arms trade as an international trade network (see \cite{Ward:2007}, \cite{WaAhAr:13}). Formally, this network can be represented as a set of countries $C^t = \{1,2, \dots, N^t\}$ varying over the years $t$ with $t \in \{t_0, \dots ,t_l\}$. Let $\mathcal{X}$ be the matrix with cell entries  $x_{ij}^t \geq0$ constituting the values of the sum of exports from country $i$ to country $j$ in a given year. As we are only interested in the absence or presence of a trade relationship, i.e. the respective decisions to export to a specific country, and to import from a specific country, we project the valued matrix onto a binary matrix $X$ by recoding all existing trade relations as one, zero otherwise, thereby $x_{ij}^t  \in \{0,1\}$. \\
The formal supply and demand model of Levine et al. (1994) is currently the most parsimonious existing theoretical framework for the exchange of arms volumes. It assumes the joint existence of economic and security motives in arms trade. These motives may sometimes clash, i.e. the economic incentives of domestic companies may collide with the security concerns of the state. We conclude from this approach that a) economic considerations and security considerations are both at play in decisions on link formation, and b) that their relative impact may change over time. For example, during the period of the Cold War with its bipolar system structure, security considerations should have had a relatively larger impact as compared to economic incentives than after the end of the Cold War. \\
In order to implement the international arms trade as a market-like exchange network we rely on the gravity equation (see \cite{WaAhAr:13}, \cite{Feenstra16} ch. 5 \& 6). The gravity equation follows Newton’s gravitation law but has an explicit economic micro-foundation. It assumes the amount of exchange $X_{ij}$ between each country $i$ and $j$ to be a function of the product of their respective economic sizes $Y$ (i.e. GDP). Geographic distance $d$ captures the cost of transactions (see \cite{Feenstra16}: 133) with $\rho$ capturing the appropriate distance elasticity. $A$ represents a constant:  
$$X_{ij} =A \cdot \frac{Y_i Y_j}{d^\rho}$$
We assume in the following that the observed binary decision to trade relevant volumes should reveal the relative impact of theoretically derived decision criteria. 
In order to allow for a mutual exchange of products in the first place, a major assumption is that companies produce differentiated products. Further developments of the gravity approach focused on the identification of unobserved price competition, of economies of scale, and of multilateral trade restrictions of both exporting and importing countries (e.g. border effects). Thus, the equation successively incorporated the sectoral home market size of a supply country, which is considered to further the production of larger number of differentiated products and allow the country to become a net exporter (\cite{Feenstra16}: 144- 147). Only efficient companies that are exhibiting higher productivity are able to survive and to expand production for exporting (ibid.) to other countries. The latter are characterized additionally by their sectoral total expenditures on the products under consideration. 
Finally, the measurement of geographic distance as a trade cost or a trade friction has been generalized as a social distance ('remoteness') and as all types of 'multilateral resistances' (i.e. tariffs, regional trade agreements, borders) from the view of both exporting and importing countries (see \cite{Feenstra16}: 168). In order to ensure a theory-consistent estimation, 'multilateral resistances' are usually specified as country fixed effects, or in the case of panel studies as country-year fixed effects\footnote{\cite{HeadMayer14} highlight the problem of estimating country attributes' impact in this case.}. \cite{WaAhAr:13} have accentuated the fact that the usual econometric gravity equation approach assumes dyadic independence. The authors show this to be theoretically questionable - and empirically refute it in their analysis of commercial trade data. 
As the exploratory description of our data has illustrated, we also have enormous clustering and transitivity in the arms trade network. Therefore, a network approach accounting for network effects is required. 
 %We deviate from the approach of \cite{WaAhAr:13} in at least two aspects. First, we will propose a statistical network approach where we will specify network dependencies explicitly. Second, we will focus on the absence or presence of a minimal positive amount of trade in order to explain the occurrence of a trade relationship rather than the value of the transaction. We intend to fully understand the decision of countries to trade arms\footnote{We will provide more details in the section 'Statistical Model'.}.
\subsection{Network Dependencies}
The reliance on \cite{Levietal:94} and on the gravity equation framework allows us to derive our hypotheses with regard to arms trade decisions.
We begin with hypotheses on the network dependency structures that a statistical network model has to control for. The implementation of so-called exponential random graph models (ERGM) in such cases usually requires the specification of an established canon (see \cite{art:Snijders2011, Cranmer-etal.17}) of endogenous dependency structures, i.e. the distribution of outdegrees, indegrees, reciprocated relations (the existence of both directions of relations within the same dyad), and transitive closures of triads. 

\begin{figure}
\centering
\includegraphics[width=\textwidth]{theory4_esp.png}
\caption{An illustration of the hypothesized endogenous effects}
\label{theory}
\end{figure}
Here, we provide explicit theory-based hypotheses for these dependence structures. \\
First, the pursuit of the economic incentive by highly productive companies with decreasing costs per unit (i.e. increasing returns to scale) should lead to market concentration at the national and international levels, i.e. to an accumulation of outdegrees by a few market leaders on the supply side (see Figure \ref{theory}, Exporter Effect). The reverse side of the coin of asymmetric concentration of exports is the dependence of most importers from these few suppliers (i.e. the occurrence of many countries with relatively low indegrees) (see Figure \ref{theory}, Importer Effect). As the distribution for indegrees in our data proved to be skewed, but to a much flatter degree, we expect the extent of the parameter to be not as negative as for outdegrees. \\
These two processes imply that the non-reciprocation of trade relations predominates, i.e. there should be much fewer mutual trade relationships than expected at random given the number of nodes. Reciprocated intrasectoral trade in arms should be the exception due to the enormous disparities in the technological development of weapons systems (see Figure \ref{theory}, Asymmetry). \\
Thus, we derive the following three hypotheses which are in accordance to what we expect from competitive economic market processes, but with more dramatic asymmetries than in commercial markets:
\vspace{-0.5cm}
\begin{itemize}
\begin{singlespace}
\item[] \textbf{Hypothesis 1:} Outdegree effect: Countries have on average fewer outdegrees than expected at random.
\item[] \textbf{Hypothesis 2:} Indegree effect: Countries have on average fewer indegrees than expected at random. 
\item[] \textbf{Hypothesis 3:} Reciprocity effect: Reciprocation of arms transfer relations between countries is lower than statistically expected.
\end{singlespace}
\end{itemize} 
The emergence of these historical realities and their structure-preserving effects are not self-evident. We will provide the first statistical estimates for these blatant structures and their historical resilience. The estimates capture the ease of access of few exporters to many national markets, and the constraints of importing countries to rely on many providers having unobserved multilateral resistances. The advantage of statistical network models as compared to econometric gravity equations is the possibility to specify these effects separately and simultaneously. \\
The econometric gravity equation also precludes the consideration of third-country effects \citep{WaAhAr:13}. A network approach allows just this. As the most frequent transitive triads in our data indicate, there seems to be a deviation from the economic principle of purely competitive markets. Triadic closure is rather known to be a feature of friendship and cooperation networks, such as in alliances \citep{Cran2012}, but not of economic actors vying to increase their own market shares. Here, the security dimension of arms trade comes into play. In order to minimize negative security externalities in the future, the formation of trust-ensuring structures proves to be important. Surveillance of and influence on the importing actor becomes easier with sharing partners. Two suppliers lower their risk when jointly sharing import countries. Transaction costs of monitoring decrease, and the latent threat of combined pressure can deter an importer to leave once agreed-upon principles. \cite{Chaney:2016} elaborated this mechanism for international trade more generally: as traders do not seem to build relations at random, it is obvious that the provision of information on the reliability and trustworthiness of other (third) traders’ 'percolates' along already existing relations.
In accordance with hypotheses 1-3 which suggest a hierarchical and centralized structure, we expect a structure of nested hierarchical triadic closures. This effect can be operationalized using a statistic called edge-wise shared partners (ESP) (see Figure \ref{theory}, last row) (see \cite{HunHan:06}, \cite{SnijPattRobHand:06}). Here, two countries connected by an asymmetric or symmetric trade relationship share many other common trading partners. This dynamic can be summed up with the following hypothesis:
\vspace{-0.5cm}	
\begin{itemize}
\begin{singlespace}
\item[] \textbf{Hypothesis 4:} Nested Triadic Closure/ESP: Supply countries with trade relations among themselves have a tendency to share many importing partners. 
\end{singlespace}
\end{itemize}
The final endogenous mechanism we consider is strong path dependency (see Figure \ref{theory}, Path Dependency). Due to the availability of only a few suppliers, previous arms transfers should lead to a pronounced inertia in trade relations. 
Additionally, the transaction costs associated with switching are higher the more trade has been accomplished in previous transactions. 
Trust in the trading partner (from the perspective of both importer and exporter), the reputation of the exporter (with regard to quality and reliability), the reputation of the importer (solvency, security externalities), and interoperability of new and old stocks render changing trade partners costly. This leads to a so-called lock-in between supplier and importer countries.
\vspace{-0.5cm}
\begin{itemize}
\begin{singlespace}
\item[] \textbf{Hypothesis 5 (Path Dependency):} 
We expect a high tendency for established trading links to persist.
\end{singlespace}
\end{itemize}  


%We have tried to provide substantial rationales for the inclusion of these network effects. %From a statistical point of view they ensure unbiased results for country-level (e.g. GDP, military capabilities) and relational effects (alliance relation, political similarity etc.) which will presented in the next subsection.


\subsection{Hypotheses Related to Market Sizes and Transaction Costs in the Gravity Equation}

As discussed above, the gravity equation accentuates the impact of market sizes and transaction costs between pairs of countries. From a network perspective, these factors relate to attributes of countries (e.g. GDP of a country) and to relational attributes (e.g. the geographic distance between two countries). In order to identify the push and pull effects of the sizes of markets of a pair of countries, we propose hypotheses specific to exporters and importers. First, in line with recent developments of the gravity approach, we consider the overall size of a domestic economic market as a precondition for the formation of productive companies that are able to export and to compete in the international market. From the perspective of importers, the availability of economic resources is necessary for making weapons payments. To account for the domestic absorption of the production of MCWs, we consider also domestic national material capabilities \citep{Singer1972} a driver for the provision of and the demand for such arms. Moreover, the conflict involvement of a country should also increase its demand for weapons: 

Accordingly, we formulate the following hypotheses capturing the economic incentives of the political economy model of international arms supply and demand: 
\vspace{-0.5cm}
\begin{itemize}
\begin{singlespace}
\item[] \textbf{Hypothesis 6 (Size of the Domestic Economic Market and of the National Military Capabilities):}
\begin{enumerate}[label=\alph*)] 
\item The larger the size of the domestic economic market, the higher the probability of observing the export of arms.  
\item The larger the size of the domestic economic market, the higher the probability of observing the import of arms.   
\item The greater the domestic national material capabilities of a country, the higher the probability of observing the export of arms.   
\item The greater the domestic national material capabilities of a country, the higher the probability of observing the import of arms.  
\item Conflict involvement of a country leads to an increase in the probability of it importing arms. 
\end{enumerate}
\end{singlespace}
\end{itemize}
In accordance with the gravity equation, the final type of effect we must consider are relational features increasing or decreasing the costs of weapon transactions between a pair of countries. Transaction costs not only arise due to geographic distance but there are additional factors of 'remoteness' and of social distance. Sharing a membership in the same formalized security arrangements, as well as commitment to similar procedures of collective decision-making and involvement of civil society are indicative of similar political values. We therefore expect that joint membership in defense agreements and regime similarity will facilitate the bilateral flow of arms between two countries. Accordingly, we formulate the following hypotheses, with 7b and 7c capturing the security incentive of the political economy model of international arms trade: 
\vspace{-0.5cm}
\begin{itemize}
\begin{singlespace}
\item[] \textbf{Hypothesis 7 (Transaction Costs):} 
\begin{enumerate}[label=\alph*)]
\item The smaller the geographic distance between two countries, the larger the probability of the presence of arms transfers between them.   
\item Joint membership in defense agreements increases the probability of arms transfers between countries.   
\item The more similar the political regimes of two countries, the higher the probability of arms transfers between them.   
\end{enumerate}
\end{singlespace}
\end{itemize}

Lastly, we intend to empirically disentangle the value trade-off between security and economic incentives intertemporally. Following the conjectures that the economic incentive increases in importance after the Cold War and the financial crisis 2007/8, we expect an increase in the effect of the economic incentive relative to the security considerations.





\section{Statistical Model}\label{statistical_model}
In this chapter we describe our statistical model and the operationalization of concepts. 

%The networks vary in size from year to year as quite a few states enter the system during the period of observation, and a smaller number exit the system. 
%Our statistical methodology is able to adjust for the entry and exit of states from one time period to the next as explained below. 
%We then subject this series of arms trade networks to a statistical analysis using state-of-the-art methodology for statistical inference on highly interdependent network data. 
 
 \subsection{Statistical Analysis of Dynamic Networks}\label{san} % approx 5-6 para
 
%Network analysis is a fast growing interdisciplinary research field with contributions coming from e.g. sociology, economics, business studies, political science, physics, computer science and statistics \citep{Jackson2008, EastKlei:10, Newm:10}. \cite{book:Kolaczyk2009} and \citet{Lushetal:13} give a first comprehensive collection of statistical contributions in the field. %\citet{art:Snijders2011} provides a survey of recent statistical models for social networks.
In the following, we introduce an approach for the modeling of the asymmetric binary decision to trade arms. 
A wave of methodological work has recently made clear that dyadic international relations data cannot be analyzed with traditional methods such as logistic regression \citep{Ward:2007, WaAhAr:13, art:Thurner2009, art:Cranmer2011}. 
The reason regression methods may not be applied is that such methods assume the conditional independence of observations. 
\cite{art:Cranmer2011} shows that the assumption of conditional independence is inextricable from models based on the regression framework. 
To give the assumption of conditionally independent observations substantive context, \cite{Cranmer2012a} point out that assuming conditional independence in a traditional model of conflict is akin to assuming that the British declaration of war on Germany in 1939 was entirely unrelated to the German invasion of Poland, conditional on predictors like joint democracy and capability ratios. 
As such, we require a statistical technique that can account for network dependencies while modeling the binary and asymmetric decision to trade arms. \\
\indent To satisfy this need, we apply the exponential random graph model (ERGM) and its extension for longitudinal data, the temporal ERGM (TERGM) \citep{HaFuXi:10}. For the modeling of longitudinal data, there are possible alternatives like the stochastic-actor based approach by \cite{Snijders:17}, the latent space approach by \cite{Hoff_etal.2002}, and the bilinear autoregression model by \cite{Minhas_etal.2016}. TERGMs are a simple, efficient and valid tool for modeling large networks that exhibit strong structural inertia, have complex nested triadic structures, and change over discrete time periods (i.e. yearly). For most suppliers, geographically distant structures cannot be conceived as local structures. Therefore, a utility-maximizing perspective is much less adequate than the tie-related perspective that is inherent to ERGMs (see \cite{Block-etal.16}). Our focus on higher-order nested triadic structures also currently precludes the usage of the approach proposed by \cite{Minhas_etal.2016}.\\
% Intuitive introduction to the ERGM and TERGM
The ERGM was introduced by \citet{art:Wasserman1996} and has since been developed into a prominent technique for modeling networks. 
It can accommodate exogenous predictors (standard covariates) at the node (countries, in our case) or link (dyadic) levels, as well as nearly arbitrary endogenous dependencies as statistics computed on the network. 
Exogenous predictors are important for testing our hypothesized effect related to countries and their relations. 
Endogenous dependencies are especially important in our case because our theory centers around five dependencies endogenous to the arms trade network (see Figure \ref{theory}). 
The ERGM enables us to operationalize and include these dependencies along with the set of state- and dyad-level controls suggested by the literature. 
The temporal extension to this model allows us not only to compute coefficients pertinent to multiple years, but also to examine the temporal heterogeneity in the magnitude of effects. 

% Technical details on ERGM
Consider a network $y$ with $N$ nodes (here, countries) as a $N\times N$ adjacency matrix with entries $y_{ij} = 1$ if nodes $i$ and $j$ are connected by a tie and $y_{ij} = 0$ otherwise. 
In ERGMs, one assumes that network ties are a function of network statistics such that 
\begin{equation*}
\label{eq:edgesundir}
P_{\theta}(Y=y) = \frac{\exp \bigl\{\theta’ s(y) \bigr\}}{\sum_{y^* \in \mathcal{Y} }exp\bigl\{\theta’s(y^*)\bigl\}},
\end{equation*}
where $s(y)$ is a vector of network statistics, $\theta$ is the vector of parameters of interest and $\mathcal{Y}$ is the set of all possible networks on $N$ nodes. 
The vector of statistics $s(y)$ is where the ERGM gets most of its power: this vector includes state-level covariates, dyad-level covariates, and network dependencies. 
This vector is where we operationalize the endogenous dependencies that correspond to our hypotheses and also control for the effects suggested by the existing literature on arms trade. 
  
% Technical details on TERGM
While ERGMs were originally developed to account for dependencies in cross-sectionally observed networks \citep{art:Snijders2002}, more recent developments propose dynamic models which allow network structures to change over time \citep{HaFuXi:10}. 
\cite{HaFuXi:10} introduced the temporal exponential random graph model (TERGM), which is specified as 

\begin{footnotesize}
\begin{equation*}
P_{\theta}(Y^t=y^t~| ~Y^{t-1}=y^{t-1}, \dots , Y^{t-k}=y^{t-k} )= \frac{exp\bigl\{\theta’s(y^t, y^{t-1}, \dots , y^{t-k})\bigl\}}{\sum_{y^* \in \mathcal{Y}^t }exp\bigl\{\theta’s(y^*, y^{t-1}, \dots , y^{t-p})\bigl\}}.
\end{equation*}
\end{footnotesize}

\noindent Note that the only real change in the model is that the vector of statistics $s(\cdot )$ can include functions of the network at time $t$ just as in the cross-sectional ERGM in equation \eqref{eq:edgesundir}, but also temporal statistics such as the stability of a tie, delayed reciprocity and so forth. 
In other words, the model for the network at time $t$ is conditioned on some number $k$ of previous realizations of the network. 
Note too that the number of nodes and therefore the dimensionality of $\mathcal{Y}$ can change over time. 
This is necessary, since states exit (e.g. USSR) and other states enter (e.g. Slovakia) the trade network.
For our primary analysis, we compute the TERGM for each year in our period of observation, conditioned on the four previous years. 
Note that in order to condition on the networks of the previous four years, all networks have to be defined on the same set of nodes. 
Thus, the TERGM for the time period $t-4,\dots , t$ includes all countries in the system at time $t$. 
We estimate our models using the \textbf{xergm} \citep{Rxergm} package in the \textbf{R} statistical environment. The estimation is based on maximum pseudolikelihood with a bootstrap correction to the confidence intervals (see \cite{LeiCraDes:17}).


 
 
\subsection{Operationalization of Effects}\label{one} % 4-5 para

Our theory posits the following five endogenous network effects -- importer effects, exporter effects, reciprocity, edgewise shared partners, and path dependency -- and a preference for node-similarity (homophily) with respect to regime type and joint membership in defense alliances, as well as an economic attraction by large importer markets. Each network effect is operationalized as a statistic computed on the network and included in the $s(y)$ vector of the TERGM. 

% out-degree
The simplest network effects to account for are exporter and importer effects. 
For exporters, our theory indicates that we should see few states exporting to many countries. 
The outdegree effect is included in our TERGM specification by computing a statistic that accounts for the outdegree of all vertices in the network. 
\citet{HunHan:06} propose geometrically down-weighting endogenous statistics to enable the inclusion of a statistic's entire distribution into the model and to help avoid degeneracy in estimation. 
Therefore, we include the geometrically weighted outdegree (\emph{gw-outdegree}). 
Because there should be fewer states that export at all, and even fewer who export to many importers, we expect a negative coefficient for the \emph{gw-outdegree} parameter. 

% in-degree
Exactly the opposite of outdegree, we operationalize importer effects by considering the indegree of a node. 
Our theory predicts that most states will be importers of arms if they participate in the market at all. 
However, our theory also suggests that importers are not able to purchase from the full set of states offering arms for export. 
Importers will usually be restricted to very few exporters from which they can actually purchase. 
In most cases, this will entail arms importation from a single exporter. 
Thus, we include a term for indegree and down-weight this statistic in the same way as we did for outdegree (\emph{gw-indegree}). 
Because importing arms from a large number of exporters is expected to be a rare situation, we expect the geometrically weighted indegree parameter to be negative.


% Mutuality
Third, we include a statistic to capture reciprocal relationships, in which country $i$ exports to country $j$ while simultaneously importing from it, called \emph{mutuality}. 
The mutuality statistic counts the number of such mutual relationships present in the network. 
Because we expect mutuality only among few members of the dense core of the arms trade network, and, by definition, only among arms-producing countries, we expect a negative effect for mutual relationships overall in the network. 


% GWESP
Fourth, we need to operationalize the process that leads to a dense core and sparse periphery while accounting for ''friend-of-a-friend-is-a-friend'' dynamics. 
To do this, we would essentially count the number of closed (transitive) triangles in the network. 
However, this approach would not distinguish between the probability of the dyad $ij$ being mutually tied to two or two-hundred other states $k$. 
As such, we geometrically down-weight the $k$ triangles terms with a structural statistic called geometrically weighted edge-wise shared partners (\textit{gwesp}) \citep{HunHan:06}. 
\textit{esp(k)} is defined as the number of connected pairs in the network that are connected over exactly \textit{k} directed paths of length 2. 
Figure \ref{theory} visualizes \textit{esp(0)}, \textit{esp(1)} and \textit{esp(2)}.  
We justify the inclusion of the geometrically weighted \textit{esp} by the fact that only few (allied) countries (e.g. US and UK) with highly developed military technologies export to many common partners, 
and we anticipate a higher number of such configurations than statistically expected given the size of the network. Therefore, the estimated coefficient for \textit{gwesp} should be positive. 

% Path depende
Fifth, we add an effect for path dependency. 
We measure path dependency as the sum total TIV sold from country $i$ to country $j$ for the five years before the year of consideration. 

% Like-regime homophily and Defense Alliance
Our final nodal attribute effects are that of homophilous mixing due to joint membership in defense alliances and by regime type, as well as due to economic incentives.
% Defense agreement
To operationalize defensive alliance homophily, we include an indicator whether a pair of states has a defensive agreement. 
Data for this indicator are drawn from the \emph{Formal Interstate Alliance Dataset}, which is part of the Correlates of War (COW) Project. This data set contains military alliance agreements signed by any nation from 1815 on. We incorporate these data as symmetric adjacency matrices for each year, where a $1$ indicates that nations $i$ and $j$ signed a \textit{defense agreement}, while a $0$ denotes that the corresponding nations have not. We expect a positive impact of joint membership on the observation of a tie between two countries.\\
For regime similarity, we compute the dyad-wise absolute difference in the 21-point Polity IV scores \citep{Mars2014} between two states and include this measure as a predictor in the TERGM. 
Intuitively, when states have similar regime characteristics, this absolute difference will be small. Larger values will occur in cases where the two regimes in question are quite different. Because our theory posits that political compatibility is an essential precondition to the transfer of military power, we expect a negative effect in which arms trade becomes less likely as regimes are increasingly different.\\
% Economic incentive
\indent In order to adjust for the economic power and military capabilities of countries (home market size and external demand market size) in the network, we include the log of GDP for the exporting as well as for the importing countries, and the Composite Index of National Capability (CINC) for both the exporter and importer. 
The GDP data are drawn from \cite{Gleditsch:2002}\footnote{The Gleditsch Expanded Trade and GDP data is, to our knowledge, the only dataset that also covers socialist and communist countries prior to 1990.}.
The CINC data are drawn from the \textit{National Material Capabilities Dataset} \citep{Singer1972}. 
The CINC is a statistical measure of national power created for the COW project and conceived as a nodal attribute. It uses an average of percentages of world totals in six different components, which represent demographic, economic, and military strength. These components are: total population, urban population, iron and steel production, primary energy consumption, military expenditure, and military personnel. 
We include this data for both the importer and exporter. 

We also specify the geographic distance between countries. Including distance, along with the importer and exporter effects just mentioned, reproduces the essential form of the usual gravity model (see \cite{disdier2008}). The distance data are drawn from Gleditsch's Distance Between Capital Cities dataset\footnote{\url{http://privatewww.essex.ac.uk/~ksg/data-5.html}} and measures that distance in kilometers.

% Intra-State conflict importer
Next, we consider information about intra-state conflicts, including all episodes of civil, ethnic, communal, and genocidal violence and warfare in the importing country. 
The data come from the \textit{Major Episodes of Political Violence Project} \citep{Mars2014} and are coded on a scale of one to ten according to an assessment of the full impact of the violence of the conflict on the societies that directly experienced their effects. 
%

% Two year lag
Finally, \citet{PerkNeum:10} point out that there is a temporal lag between the order date and the delivery date for arms. 
According to SIPRI, the average time from order to delivery is approximately two years. 
Therefore, we include all covariates with a two year lag, (i.e. for the network of year $t$ we use the covariates of year $t-2$).


%%% Other established effects (pseudo controls);  ca 5 para


%%%%%%%%%%%%%%%%%%%%%%%%%%%%%%%%%%%%%%%%%%%%%%%%%%%
%%%%%%%%%%%%%%%%%%%%%%%%%%%%%%%%%%%%%%%%%%%%%%%%%%%
%%%%%%%%%%%%%%%%%%%%%%%%%%%%%%%%%%%%%%%%%%%%%%%%%%%
%%%%%%%%%%%%%%%%%%%%%%%%%%%%%%%%%%%%%%%%%%%%%%%%%%%


\section{Results}\label{results}




   

\subsection{Inferential Results}

% Why we don't pool over time
%It is common in the international relations literature to consider effects pooled over long periods of time. However, international processes may exhibit substantial temporal heterogeneity in the strength of effects over even relatively short periods of time. 
%To address this heterogeneity, we ran our TERGM, in which all covariates have a two year lag, across the period of observation with a five-year pooling period. 
We estimate a pooled TERGM for the years 1952-1956, and then another pooled TERGM for 1953-1957, and so on for the whole time period 1952-2013. The annual coefficients presented in Figure \ref{tergm1} reflect the model estimated over the previous five years. As such, the coefficient for 1956 is the estimate of the pooled TERGM from 1952 to 1956. 
This approach produces a TERGM for every year in the period of observation, each of which is temporally conditioned on the five years before it. 

% 1 paragraph introducing the temporal results and their interpretation
Figure \ref{tergm1} presents the results of our TERGM analysis of the arms trade network between 1956 and 2013, representing the endogenous (\textit{edges} (= constant) \textit{gw-indegree}, \textit{gw-outdegree}, \textit{mutuality} (= reciprocity), \textit{gwesp}, \textit{path dependency}) and the exogenous effects (node attributes and relational attributes), respectively. 
We report coefficients that are statistically significant at the traditional 0.05 level as circles, those significant only at the 0.10 level as triangles, and those that are not statistically significant at conventional levels as rectangles. 
The shaded region astride the coefficients reflects a 95\% confidence interval. 
The TERGM coefficients are interpreted as the log of the relative likelihood of observing one additional unit of the statistic under consideration (e.g. one more reciprocal dyad). 


\begin{figure}
\vspace{-3cm} 
 \begin{adjustwidth}{-2cm}{}
\begin{center}
\includegraphics[scale=.82]{tergm1}
\caption{Time series of the estimated endogenous TERGM parameters for the time period 1956-2013}
\label{tergm1}
\end{center}
 \end{adjustwidth}
\end{figure}
\renewcommand{\thefigure}{\arabic{figure} (Cont.)}
\addtocounter{figure}{-1}
\begin{figure}
\vspace{-3cm} 
 \begin{adjustwidth}{-2cm}{}
\begin{center}
\includegraphics[scale=.82]{tergm2}
\caption{Time series of the estimated exogenous TERGM parameters for the time period 1956-2013 }
\label{tergm2}
\end{center}
 \end{adjustwidth}
\end{figure} 

% 2-3 paragraphs discussing results (two on endogenous and one on exogenous)
The \textit{gw-in-} and \textit{gw-outdegree} effects reflect strong support for our first two hypotheses: exporter and importer effects. 
The consistently negative and statistically reliable effect of \textit{gw-outdegree},\footnote{We fixed the geometric decay parameter to $1$.} despite its oscillations, corroborates the expectation that there are extremely few arms exporters and that these exporters deliver to many importers. However, the long-term trend of this effect decreases implying that market concentration increases. To complement this, the \textit{gw-indegree} parameter is statistically significantly negative for most of the period of analysis, indicating that most importers are dealing with a small number of exporting states (usually only one), which clearly also reflects the discussed lock-in process. The negative effect attenuates after the 1990s, indicating the decline of alignments and trade dependencies. Compared to the oudegree effect, the indegree effect is weaker. Together, the empirical behavior we observe is strongly consistent with our theoretical claims about economies of scale and lock-in effects. The latter becomes also visible in the \textit{path dependency} effect which is consistently positive and large. 
Beyond consistent support for the inertia effect posited in our theory, it is interesting to observe the large downward shock that accompanied the collapse of the Soviet Union. 
One may understand this abrupt downturn as the result of former Soviet Block states shifting their allegiance to the West and starting to trade with NATO countries, but also selling out their old stocks to developing countries. 
Moreover, the overall trend in the effect suggests that reliance on established partnerships has been decreasing since 1960, but may have regained importance after 2001.

Support for hypothesis 3, \textit{reciprocity}, is mixed.
Reciprocity in the trading of arms would be indicative of intra-sectoral trade. 
Intra-sectoral trade could indeed be common among politically compatible producers of major weapons systems. However, intra-sectoral arms trade should be uncommon in most of the network because very few states are producers and only a small subset of those will be mutually politically compatible. 
Indeed, we see that the expected a-reciprocity is common in recent years (since 2000), as indicated by the negative mutuality effect in Figure \ref{tergm1}, but the effect is not generally reliable prior to that. Beginning around 1990 we even observe a substantial tendency for reciprocation over a period of several years.  

The \textit{gwesp}\footnote{We fixed the geometric decay parameter to $1.5$.} effect in Figure \ref{tergm1} provides strong support for our fourth hypothesis that hierarchically nested transitivity should be a major feature of the network. 
The consistent positive effect of \textit{gwesp} is an important substantive result because it indicates that the transfer of arms is in fact a form of shared risk perceptions (security criterion). The same sort of triadic effects are often found in alliance networks. The presence of such triadic closures is evidence of the network having structures supporting the military alignment of third countries. It corroborates findings \citep{Chaney:2016} that sharing trading partners is a mechanism for lowering transactions costs more generally. 

We also find clear evidence for our hypothesis that countries with similar regimes and joint membership in defensive alliances are more inclined to trade arms (Hypothesis 6). 
Figure \ref{tergm1} shows that the absolute difference in the polity score specified as a relational attribute, except for a brief bump at the end of the Cold War, is consistently negatively associated with arms trade. 
In other words, the greater the regime dissimilarity, the less likely two countries are to trade arms. 
While this result is intuitive, assuming that polity scores can serve as a valid indicator of shared values, it is nonetheless important because this is one of the key mechanisms that puts a strategic break on otherwise market-driven forces affecting the arms trade network. 
Interestingly, while the effect is always positive (as hypothesized), defensive alliances seem to be decreasingly relevant as a facilitator of arms trade. Moreover, one can clearly see the dip in this effect following the collapse of the Soviet Union, and the subsequent recovery of the effect following the expansion of NATO after 2001. 
These dynamics can be illustrated by contrasting the respective percentage changes of the odds of observing a trade relation as compared to no trade relation: in 1980 we estimate an increase of ca. 107\% of the odds for members of alliances as compared to nonmembers, holding all other variables constant. In 2000 this effect is much smaller, i.e. 19\%. Similarly, a unit increase in the regime distance leads to an 18\% decrease of the odds of trading in 1980. In 2000, this effect diminishes to a decrease of only 11\%. The importer and exporter effects with respect to GDP show that both probabilities of exporting and importing increase with economic size, suggesting that the market size argument of international trade also holds in this sector. 
This effect is less clear with the importer and exporter effects based on the national capabilities (CINC). While the exporter effect does display a tendency to be positive, indicating that states with higher capabilities are more likely to be exporters, and the importer effect does display a tendency to be negative, both of these effects experience much statistical unreliability over the period of analysis.

Several of our control variables also tell interesting stories in Figure \ref{tergm1}. 
Aside from a reduction in the effect around the collapse of the Soviet Union, intra-state conflict consistently induces higher volumes of arms importation. Interestingly, and in strong contrast to what empirical gravity models verified for other trade sectors, we see no consistently statistically reliable effect for geographic distance. This implies that the calculus for military power projections does not react to increasing transaction costs due to geographic remoteness.


\renewcommand{\thefigure}{\arabic{figure}}
\begin{figure}[t]
\begin{center}
\includegraphics[scale=.42]{DAvsGDPimp}
\caption{Estimated Value Trade-Off between Security Concerns versus Economic Incentives (red dots: significant parameters, gray area: confidence interval) }
\label{tradeoff}
\end{center}
\end{figure}

\subsection{Security versus Economy}

\noindent Joint alliance membership and the market size of importing countries exert strong positive influences on the probability of weapons transfer. We consider alliance membership as a valid proxy for the security considerations of the exporter and the importers' market size as a measure for the economic incentive. There has been much discussion in the literature about the trade-off between security and economic incentives in arms trading, but no empirical measure until now. To asses this trade-off, we compute point estimates and confidence intervals for nonlinear combinations of our parameter estimates.\footnote{We are relying on the delta method which we implemented in R.}
We are therefore able to test whether economic incentives actually increased their importance as compared to security concerns after the end of the Cold War. We find that after the 1970s there was a slight decrease of in the importance of security considerations. However, this trend ended abruptly in 2001 and security started regaining relative importance.  

\subsection{Model Validation}
\begin{figure}[t]
\begin{center}
\includegraphics[scale=.47]{roc_pr_auc}
\caption{Out-of-sample areas under the curve (AUC) over time of the preceding five years}
\label{auc}
\end{center}
\end{figure}
A very robust way to judge the fit and predictive performance of a model involves out-of-sample prediction. 
We took each annual TERGM, which has a temporal dependency on the previous four years, and used the estimates to predict the network in the following year. 
This constitutes true out-of-sample prediction because the models were ''trained'' on years $[t-4, t]$ and are used to predict $t+1$. As prediction criteria we consider the familiar receiver operating characteristic (ROC) curve and the less familiar precision-recall (PR) curve. The precision, also known as the positive predictive values, describes the percentage of correctly predicted ties in the sample of predictive ties, while the recall, also known as sensitivity, is the percentage of correctly predicted ties in the set of actual existing ties. We recommend the PR curve for rare-event binary classification over the ROC curve because it focuses on the accurate prediction of 1's as opposed to the ROC, which gives equal weight to the prediction of 1's and 0's. When the occurrence is rare, the PR curve is a much more demanding criterion than the ROC.
The resulting area under the curve (AUC) is interpreted as follows: the closer the AUC value to 1, the closer the model is to perfect predictive performance. 

Figure \ref{auc} plots the AUC for the entire period of analysis. 
As is clear from the figure, the TERGM predicts very well. 
By the ROC criteria, our model predicts out-of-sample with better than 95\% accuracy for all but a few years at the close of the Cold War. 
By the PR criteria, our model usually predicts between 40\% and 60\% of all arms trading connections out-of-sample.
Our model captures the data-generating processes well until 1990, when the collapse of the USSR leads to structural discontinuities in our predictive performance. The arms trade system seems to rewire itself after 2001, with old and emerging new powers like China structuring this market.    


Comparing the forecasts with those from a logit model without network statistics, but including the five-year aggregated previous trade volumes ('path dependency'), we see that the explicit inclusion of network statistics improves the model prediction after the end of the Cold War, where the bipolar structure caused by the superpowers definitely disappears. This is not to say that network dependencies are not at work in the Cold War period, but they do change less during that time and are therefore contained in the path dependency term. 
 




\section{Discussion}

% the lit has viewed arms trade as an exogenous and non-interdependent phenomena, we take the opposite approach
The study of the international arms trade has relied almost exclusively on exogenous predictors and the assumptions that dyadic relationships are independent from one another. 
We have taken the position that independence of observations is an unreasonable assumption, implying that the arms trade process may only be understood as a network, and that processes endogenous to that network are instrumental to its evolution. 
Here, we have put forward a network-based theory of arms trade decisions. 
Our theory integrates the market forces acting upon exporters and importers as well as the strategic security considerations can oppose those market forces. 
The network-based approach has allowed us to integrate these perspectives both theoretically and empirically to produce a systemic model of arms trade. 


% We find... endogenous matters!
Our results, taken together, demonstrate that not only are the previously ignored endogenous processes on which we focus relevant for the arms trade network process, but predominantly so. With the exception of reciprocity, each of the endogenous determinants of the network provides substantively important, statistically reliable, and predictively valuable results. %Additional analyses clearly show that our endogenous network effects contributed more to out-of-sample predictive fit than the exogenous effects on which the literature has focused until now\footnote{See the Appendix for a comparison of the forecasting performance of a conventional logit model without dependence structures.}. 
Our results show that analyses of arms trade activity that omit endogenous determinants of the network's structure cannot hope to be free of omitted variable bias.


% "The analysis also addresses the puzzle of"... the tradeoff between market forces and strategy
Our analysis also illuminates the inherent tension between economic and security forces in the data generating process. We reveal the relative impact of economic versus security incentives as specified in formal models of international arms trade. It turns out that alliances continue to be a major determinant of arms exchange despite a remarkable temporary decrease in the 1990s. Taking the market size of the importer (GDP) as a proxy for the economic incentive to export, we cannot identify an important increase of this criterion in exporters’ decision-making after the Cold War. A systematic comparison of the estimated value trade-off supports this result. However, we do find evidence of a temporary relaxation of the security criterion - at least until 2001, when strategic considerations regain relative importance. Thus, it seems that 9/11 could have played a decisive role in the restructuring of a new international order in arms trade. 

% The third issue... temporal heterogeneity in effects  
Our study also raised a methodological issue with implications for the study of international politics reaching far beyond arms trade: each of the effects included in our model, whether a feature effect or control, exhibited a dramatic degree of temporal heterogeneity with respect to the strength and even direction of effects. It is common practice in international relations scholarship to pool over long periods of time, often hundreds of years. Our study has shown it is unrealistic to expect the sort of temporal stability required for such analyses with respect to arms trade. 
Conceiving of arms trade as a network phenomenon represents a fundamental departure from established scholarship on the topic. This holistic approach is able to unify the arms trade literature, both theoretically and empirically, while making more credible assumptions about the nature of the data generating process. 
 %The ability to study the endogenous determinants of these complex networks in concert with the exogenous opens new horizons for scholarly inquiry. We have taken what with think is an important first step in exploring the complex processes of international arms trade and this new meso-level class of processes. 

   
\newpage
\section*{Bibliography}
\small
\bibliographystyle{chicago}
 \nocite{Squaetal:11}
 \nocite{Squaetalb:11}
 \nocite{ArmKor:14}
 \nocite{Arvietal:13}
  \nocite{Hane:84}
   \nocite{Chan:14}
\bibliography{Literatur-GK}

\end{document}
